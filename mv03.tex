\documentclass{article}
\usepackage{amsmath}

\begin{document}

\section{PV + BESS Model v3}

In this model, the solar energy generated is used to compensate the electricity demand during day and the excess solar energy is used to charge the battery during daytime.
The battery is discharged during peak night hours. Thus, the capacity of the battery is defined as,
\[
C_{\text{batt}} = ( \epsilon - 1 ) E_{\text{load}}       \tag{8}
\] 
Daily electricity bill in this case will be,
\[
B_{v3} = R_p P_p T_{pn} - (C_{\text{batt}}) R_p
\]

Savings in this case will be,
\[
S_{v3} =  P_p R_p T_{pd} + R_N P_N T_N + ( \epsilon - 1 ) E_{\text{load}} R_p   \tag{9}  
\]

For installation cost in this case, we will have to account for solar PV as well as battery. Let installation cost for battery be \( J \) Rs./kWh.
SPP in this case can be calcuated as,
\[
Y_{v3} = \frac{(K \times P_{\text{solar}}) + (J \times C_{\text{batt}})}{(P_p R_p T_{pd} + R_N P_N T_N + ( \epsilon - 1 ) E_{\text{load}} R_p) \times d_y}    \tag{10}   
\]

\subsection{Example 3.1}
Let \( J = 29000\) Rs./kWh. Let everything else be the same as Example 2.1 and Example 1.1
\[
E_{\text{load}} = 62 kWh       
\] 
\[
C_{\text{solar}} = 1.25 \times 62 = 77.5 kWh
\]
\[
P_{\text{solar}} = \frac{77.5}{5} = 15.5 kW       
\]
\[
C_{\text{batt}} = 0.25 \times 62 = 15.5 kWh       
\]
Now, using equation (9) and (10), we can find SPP,
\[
Y_{v3} = \frac{(54000 \times 15.5) + (29000 \times 15.5)}{(5 \times 5.5 \times 3 + 4 \times 3 \times 9 + 0.25 \times 62) \times 365} 
       = 17.1 years      
\]

\end{document}
