
\section{Off-Grid, Solar-BESS Model v3}
This is an extreme case of assuming unlimited area available to
install solar panels so that it is not necessary (or possible) to use
Grid energy at all, eg. as in a remote location. Let $E_\text{sp}$ be the
average energy available from one solar panel per day and $N_\text{sp}$ be
the number of solar panels installed. The duration of solar availability
is $T_\text{s}$ hr per day.

\paragraph{Assumptions}
\begin{enumerate}
    \item Though there is no peak and normal load or rates in an off-grid strategy,
    we will use the same notation as in the previous models to obtain the
    load profile and compute the savings in comparison to the current electricity bill
    ($B_\text{now}$). 
    \item The day is divided into: Half peak load during solar availability,
    normal load during solar availability, 
    half peak load during night, normal load during night time, and an idle time 
    when there is no significant load ($P_\text{n} \gg P_\text{idle}\approx 0$).
    \begin{equation}
        24 = \left[ \frac{T_\text{p}}{2} + T_\text{nd} \right]_\text{day} 
        + \left[ \frac{T_\text{p}}{2} + T_\text{nn} + T_\text{idle} \right]_\text{night}
    \end{equation}
    \item During day time BESS is charged. This means solar
    energy is used for the load as well as charging BESS sufficiently
    for the night time discharge.

\end{enumerate}

\paragraph{Capacity of the BESS system} We have to revise the capacity from the
earlier estimate as the BESS now has to support a larger load.  BESS
discharges only during the night time supporting the half peak load and normal night load. 

\begin{equation}
    C = \frac{P_\text{p} T_\text{p} }{2 \eta} + \frac{P_\text{n} T_\text{nn} }{\eta}.
\end{equation}
During the day time, solar supports half of "peak" load, normal day-time load,
and charges the BESS.  Solar capacity needed is
\begin{equation}
    \label{eq:cs}
   C_\text{solar} = 
        \frac{P_\text{p} T_\text{p} }{2} + P_\text{n}  \left[T_\text{s} - \frac{T_\text{p}}{2} \right] + C
\end{equation}
The number of solar panels required is
\begin{equation}
    N_\text{sp} = \frac{C_\text{solar}}{E_\text{sp}}.
\end{equation}
The daily electricity bill for the off-grid system is
\begin{equation}
    B_\text{v3} = R_\text{s} \, C_\text{solar}.
\end{equation}
The savings due to the Solar-BESS is
\begin{equation}
    S_\text{v3} = B_\text{now} - B_\text{v3} 
\end{equation}
There is no simple expression for the payback period in this case.
\begin{equation}
    Y_\text{v3} = \frac{K \, C}{S_\text{v3} \, d_\text{y}}.
\end{equation}
and the payback period is to be calculated numerically step-wise, dependent
on all the parameters used BESS capital cost, power rates, load 
consumption pattern, solar availability, and round-trip efficiency.

\paragraph{Example 3.1} Off-grid Solar-BESS system.
Let $R_{\text{s}} = 2$ Rs./kWh, $R_\text{n} = 7$Rs./kWh, $\alpha=0.25$,
$T_\text{p}=8$~hrs, $P_\text{p} = 40$~kW, $T_\text{nd}=8$~hrs, $T_\text{nn}=0$,
$P_\text{n} = 50$~kW, $T_{\text{s}} = 7$~hrs, $E_{\text{sp}} = 4.5$ kWh, $K =
30000$ Rs./kWh, $\eta = 0.95$. 

The daily load is
\begin{align}
    E_{\text{load}} & = P_{\text{p}} T_{\text{p}} 
    + P_{\text{n}} T_{\text{nd}} + P_\text{n} T_\text{nn} 
    + \cancelto{0}{P_\text{idle}} T_\text{idle}
    \\
    & = 40 \times 8 + 50 \times 8 + 0 = 720 \text{ kWh}.  
\end{align}
Energy bill without solar+BESS is
\begin{align}
    B_{\text{now}} & = 
    R_\text{p} \, P_{\text{p}} T_{\text{p}} + R_\text{n} ( 
    P_{\text{n}} T_{\text{nd}}+ P_\text{n} T_\text{nn} )   \\
    &= 8.75 \times 320 + 7 \times 400 = \text{Rs. } 5600
\end{align}
BESS capacity is
\begin{align}
    C & = \frac{P_\text{p} T_\text{p} }{2 \eta} + \frac{P_\text{n} T_\text{nn} }{\eta} \\
    & = \frac{40 \times 8}{2 \times 0.95} + 0 = 168 \text{ kWh} \approx\to 180 \text{ kWh}.
\end{align}
Using battery modules in units of approx 15 kWh, we need ceil(168/15) = 12 modules, or 180 kWh.
The solar capacity is
\begin{align}
    C_{\text{solar}} & = 
    \frac{P_{\text{p}} T_{\text{p}} }{2} + P_{\text{n}} \left[T_{\text{s}} - \frac{T_{\text{p}}}{2} \right] + C
    = 40 \times 4 + 50 \times 3 + 180 = 490 \text{ kWh}.
\end{align}
Number of solar panels required is
\begin{equation}
    N_{\text{sp}} = \frac{490}{4.5} \approx 109.
\end{equation}
Daily energy bill with Solar-BESS is
\begin{equation}
    B_{\text{v3}} = 2 \times 490 = 980 \text{ Rs.}
\end{equation}
Daily savings is
\begin{equation}
    S_{\text{v3}} = 5600 - 980 = 4620 \text{ Rs.}
\end{equation}
Capital cost of the BESS is
\begin{equation}
    K = 30000 \text{ Rs./kWh} \times 180 \text{ kWh} = \text{Rs. }54 \text{ lakhs}.
\end{equation}
Payback period is
\begin{equation}
    Y_{\text{v3}} = \frac{54 \text{ lakhs}}{4620 \times 300} \approx 4 \text{ years}.
\end{equation}

%%% Local Variables:
%%% mode: latex
%%% TeX-master: "paybackPeriod"
%%% End:
