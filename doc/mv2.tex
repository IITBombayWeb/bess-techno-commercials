\section{BESS Peak-shifting Solar Charging Model v2}
In this model the BESS is charged using solar power during the day, and
discharged during the peak-rate hours. The solar power is
assumed to be available at the rate of $R_{\text{s}}$~Rs./kWh. Though
there is no actual running cost (other than maintenance), the solar
rate is essentially the capital cost ammortized over the life of the
solar panels.  The electricity bill in this case consists of the normal rate 
and using the solar power to charge the BESS sufficient to meet the 
peak power demand.
\begin{equation}
    B_{\text{v2}} = 
    R_{\text{n}} P_{\text{n}} T_{\text{n}} +
    R_{\text{s}} (2 \, C)
\end{equation} 

Daily savings due to BESS is
\begin{equation}
    S_\text{v2} = \left(R_{\text{p}} - \frac{R_{\text{s}}}{\eta}\right) P_{\text{p}} T_{\text{p}} 
    = \left(R_{\text{p}} - \frac{R_{\text{s}}}{\eta}\right) \, 2 \, \eta \, C \, .
\end{equation}
The payback period is
\begin{equation}
    Y_\text{v2} = \frac{K \, C}{S_\text{v2} \, d_\text{y}} 
    = \frac{K}{2 \, \eta \, d_\text{y}} \frac{1}{\left(R_{\text{p}} - \frac{R_{\text{s}}}{\eta}\right)}.
\end{equation}
Note that the expression for the paypack period is independent of the
actual consumption, total captital cost, and capacity of the BESS. It
only depends on the capital cost per kWh, effective solar rate,
the round-trip efficiency, and the peak rate (in contrast to 
Model v1, which depended on normal rate).


\paragraph{Example 2.1} Payback period for a BESS with solar charging. 
Let $R_{\text{s}} = 2$ Rs./kWh, and the rest same as Example 1.1.
The payback period is
\begin{equation}
    Y_\text{v2} = \frac{30000}{2 \times 0.95 \times 300} 
    \frac{1}{8.75 - 2/0.95} \approx 8 \text{ years}.
\end{equation}


%%% Local Variables:
%%% mode: latex
%%% TeX-master: "paybackPeriod"
%%% End: