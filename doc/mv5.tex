\section{Combustion Engine (Diesel generator) replacement BESS Model v5}

In this model, the BESS is used to replace a diesel generator or any other
fuel-combustion engine such as that using LNG. For simplicity, we will
assume that the generator is used only during the peak hours, when there
is likely to be a power cut. BESS is used to supply the load during the
peak hours and is charged during the normal hours (BESS peak shifting).

Let $R_{\text{g}} = (1 + \gamma) \, R_\text{n}$ be the rate (Rs./kWh) for the
generator fuel, and $T_{\text{g}} = \tau_{\text{d}} \, T_\text{p} $ be the duration of
the generator operation per day. Here, $\tau_\text{d}$ is the fraction of the
peak period there is a power-cut and the generator supplies the load.  The daily
electricity bill for the generator supported backup power for peak hours is
composed of the normal rate, peak rate without the generator, and generator
rate:
\begin{equation}
    B_{\text{g}} = 
    R_{\text{n}} P_{\text{n}} T_{\text{n}} +
    R_{\text{p}} P_{\text{p}} (T_{\text{p}} - T_\text{g})  + 
    R_{\text{g}} P_{\text{p}} T_{\text{g}} .
\end{equation}
In Model v5, the generator is then replaced by the BESS during the peak hours (as a power
backup as well as peak-shifting). This means BESS takes over the load during
peak, irrespective of whether or not there is a power-cut. It is charged during
the normal hours.  In this case the daily electricity bill is (as with Model v1
in Eq.~\eqref{eq:bv1})
\begin{equation}
    B_{\text{v5}} = 
    R_\text{n} \left( P_{\text{n}} T_{\text{n}} +  
    2 C \right)
\end{equation}
where $C$ is the capacity of the BESS required to support the peak power demand.
The savings due to the BESS peak-shifting is
\begin{align}
    S_\text{v5} = B_{\text{g}} - B_{\text{v5}} 
    & = R_{\text{p}} P_{\text{p}} (T_{\text{p}} - T_\text{g})  + 
    R_{\text{g}} P_{\text{p}} T_{\text{g}} - 2 R_\text{n} C \\
    & = R_\text{n} \left[ (1 + \alpha) P_{\text{p}} (T_{\text{p}} - T_\text{g})  + 
    (1 + \gamma) P_{\text{p}} T_{\text{g}} - 2 C \right]\\
    & = R_\text{n} \left[ (1 + \alpha) 2 \eta \, C + 
    (\gamma - \alpha) P_{\text{p}} T_{\text{g}} - 2 C \right] \\
    & = R_\text{n} \left[ (1 + \alpha) 2 \eta \, C + 
    (\gamma - \alpha) (2 \, \eta \, C) \tau_{\text{d}} - 2 C \right]\\
    & = 2 R_\text{n} \, C \left[ \eta \, (1 + \alpha) +
    \eta \, \tau_{\text{d}} \, (\gamma - \alpha) -  1 \right]. \label{eq:sv5} 
\end{align}
Note that Eq.~\eqref{eq:sv5} reduces to Eq.~\eqref{eq:sv1} when $\tau_{\text{d}} = 0$.
The payback period is
\begin{equation}
    Y_\text{v5} = \frac{K \, C}{S_\text{v5} \, d_\text{y}} 
    = \frac{K}{2 \, R_\text{n} \, d_\text{y}} \frac{1}{\left[  \eta \, (1 + \alpha) +
    \eta \, \tau_{\text{d}} \, (\gamma - \alpha) -  1 \right]}. \label{eq:pb5}
\end{equation}
Rewriting the above equation for $\tau_{\text{d}}$ in terms of the payback period,
we get
\begin{equation}
    \tau_{\text{d}} = \frac{1}{\eta \, (\gamma - \alpha)} \left[ 
    \frac{K}{2 \, R_\text{n} \, d_\text{y} \, Y_{\text{v5}}} + 1 
    - \eta \, (1 + \alpha) \right].
\end{equation}

\paragraph{Example 5.1} Payback period for a BESS replacing a diesel generator.
Let $R_{\text{n}} = 7$ Rs./kWh, $\alpha = 0.25$, $\gamma = 3$, $K=30000$ Rs./kWh,
$\eta = 0.95$, $T_{\text{g}} = 1 $~hr, $T_\text{p}=8$~hr. The payback period is
\begin{equation}
    Y_\text{v5} = \frac{30000}{2 \times 7 \times 300} 
    \frac{1}{[0.95 \times 1.25 + 0.95 \times (1/8) \times (3 - 0.25) - 1]} 
    \approx 13 \text{ years}.
\end{equation}

\paragraph{Example 5.2} For how many hours of generator use per day, will the BESS
payback in a period of 5 years?

The fraction of time the generator is used is
\begin{equation}
    \tau_{\text{d}} = \frac{1}{0.95 \times (3 - 0.25)} \left[ 
    \frac{30000}{2 \times 7 \times 300 \times 5} + 1 
    - 0.95 \times 1.25 \right] \approx 0.48.
\end{equation}
Or BESS can replace cases where the generator is being used for at least 
$T_\text{g} \approx 4$ hours per day.

%%% Local Variables:
%%% mode: latex
%%% TeX-master: "paybackPeriod"
%%% End:
