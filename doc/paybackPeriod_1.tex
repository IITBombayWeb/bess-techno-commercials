\documentclass[a4paper]{article}
\usepackage{amsmath}
\usepackage{palatino}
\usepackage{mathpazo}
\usepackage{calrsfs}

\usepackage{booktabs}
\usepackage{graphicx}
\usepackage{bm}
\usepackage{mathtools}
\usepackage{datetime}
\mathtoolsset{showonlyrefs}
\pagestyle{empty}


\begin{document}

\title{Payback Period of BESS Strategies}
\author{P Sunthar}

\maketitle


\begin{tabular}{ll}
Memo no. & \texttt{PS-0001T\_RevA} \\
Date & \today \\
To & GN, KC, MR, BW, CK \\
From & PS \\
Subject & Payback Period of BESS Strategies \\
\end{tabular}


\section{Introduction}

The payback period (PP) of a battery energy storage system
(BESS) is a key metric for evaluating the economic feasibility of a
BESS project. 

Instead of doing a rigorous economic analysis, we will focus on the key
factors that determine the payback period of a BESS. We will use simple models
for the capital cost and savings of a BESS, and derive the PP under
various strategies of operation.


\section{BESS Peak-shifting Model v1}

\paragraph{Peak-shifting} This is the most common strategy for BESS
and means that the battery is charged during off-peak hours and 
discharged during peak hours. The objective is to reduce the peak
power demand and thereby reduce the electricity bill.

\paragraph{Assumptions}
\begin{enumerate}
    \item Grid power is available at two prices: $R_{\text{p}}$ for peak
    rate (Rs./kWh) and $R_{\text{n}}$ (Rs./kWh) for normal or base rate.
    The relationship is defined
    \begin{equation}
        R_{\text{p}} = (1 + \alpha) R_{\text{n}},
    \end{equation} 
    where $\alpha$ is the premium for peak-rate power.

    \item The peak-rate period per day is $T_{\text{p}}$ hours, and the 
    normal-rate duration is $T_{\text{n}}$ hours per day.
    $T_{\text{n}} > 2 T_{\text{p}} $, so that there is sufficient time 
    to charge the battery at 0.5~C at normal rate. 

    \item During peak-rate hours, the power demand is $P_{\text{p}}$ kW, 
    and during normal-rate hours, the power demand is $P_{\text{n}}$ kW.

    \item The BESS has a capacity of $C$ kWh, and the round-trip efficiency 
    is $\eta$.
    The capacity of the BESS is chosen such that the peak power demand is met. 
    The capacity required is
    \begin{equation}
        C = \frac{P_{\text{p}} T_{\text{p}} }{2 \, \eta}.
    \end{equation}
    The factor 2 accounts for the fact that the battery charged/discharged twice a day,
    because the peak-rate period is spaced out over two parts of the day.


\end{enumerate}

\paragraph{Model}

Without the BESS, the daily electricity bill (in Rs.) is
\begin{equation}
    B_{\text{now}} = 
    R_{\text{n}} P_{\text{n}} T_{\text{n}}  +
    R_{\text{p}} P_{\text{p}} T_{\text{p}} . 
\end{equation}
With the BESS, grid is used only during the normal-rate hours, to supply the load
as well as to charge the battery (up to the capacity requried for the peak discharge). 
The daily electricity bill is
\begin{equation}
    B_{\text{v1}} = 
    R_\text{n} \left(P_{\text{n}}  T_{\text{n}}  +  
     C \right)
    = 
    R_\text{n} \left( P_{\text{n}} T_{\text{n}} +  
    P_{\text{p}} T_{\text{p}} \frac{1}{\eta} \right)
\end{equation}
Daily savings due to BESS is
\begin{align}
    \label{eq:sv1}
    S_\text{v1} & = B_{\text{now}} - B_{\text{v1}} = 
    R_{\text{n}} P_{\text{p}} T_{\text{p}} \left(1 + \alpha - \frac{1}{\eta}\right) \\
    & =  2 \eta \, C \,  R_{\text{n}} \left(1 + \alpha - \frac{1}{\eta}\right).
\end{align}


% For an ideal BESS ($\eta = 1$), the savings is
% \begin{equation}
%     S_\text{v1.1} = R_{\text{n}} P_{\text{p}} T_{\text{p}} \alpha.
% \end{equation}

\paragraph{Payback period} The capital cost of the BESS is $K$ Rs./kWh
of capacity. The payback period is the time taken for the savings to
equal the capital cost. The payback period (in years) is
\begin{equation}
    \label{eq:pb1}
    Y_\text{v1} = \frac{K \, C}{S_\text{v1} \, d_\text{y}} 
    = \frac{K}{2 \, \eta \, R_\text{n} \, d_\text{y}} \frac{1}{\left(1 + \alpha 
    - \frac{1}{\eta}\right)}
\end{equation} 
where $d_\text{y}$ is the number of active power consumption days in a year. 
Note that the expression for the paypack period is independent of the
actual consumption, total captital cost, and capacity of the BESS. It
only depends on the capital cost per kWh, premium for peak-rate power,
the round-trip efficiency, and the normal rate.

\paragraph{Example 1.1} Payback period for a BESS.
Let $R_{\text{n}} = 7$ Rs./kWh, $\alpha = 0.25$,
$T_{\text{n}} = 16$ hours, $T_{\text{p}} = 8$ hours, $P_{\text{p}} = 50$~kW,
$d_\text{y} = 300$ days, $K = 30000$ Rs./kWh, $\eta = 0.95$. The capacity of the BESS is 
\begin{equation}
    C = \frac{8 \times 50}{2 \times 0.95} \approx 210 \text{ kWh}.
\end{equation}
The payback period directly evaluated from Eq.~\eqref{eq:pb1} is
\begin{equation}
    Y_\text{v1} = \frac{30000}{2 \times 0.95 \times 7 \times 300} 
    \frac{1}{1.25 - 1/0.95} \approx 38 \text{ years}.
\end{equation}

\paragraph{Example 1.2} What does the peak-rate premium need to be for a payback period of 5 years. 
Approximating $\eta \approx 1$ in Eq.~\eqref{eq:pb1}, we have
\begin{equation}
    Y_\text{v1} = \frac{K}{2 \, \alpha \, R_\text{n} \, d_\text{y}} 
\end{equation}
Or the premium should be
\begin{equation}
    \alpha = \frac{K}{2 \, Y_\text{v1} \, R_\text{n} \, d_\text{y}} = \frac{30000}{2 \times 5 \times 7 \times 300}  \approx 1.4.
\end{equation}
The peak-rate should be 
\begin{equation}
    R_{\text{p}} = (1+1.4) \times 7 = 16.8 \text{ Rs./kWh}.
\end{equation}



\section{BESS Peak-shifting Solar Charging Model v2}
In this model the BESS is charged using solar power during the day, and
discharged during the peak-rate hours. The solar power is
assumed to be available at the rate of $R_{\text{s}}$~Rs./kWh. Though
there is no actual running cost (other than maintenance), the solar
rate is essentially the capital cost ammortized over the life of the
solar panels.  The daily electricity bill is
\begin{equation}
    B_{\text{v2}} = 
    R_{\text{n}} P_{\text{n}} T_{\text{n}} +
    R_{\text{s}} P_{\text{p}} T_{\text{p}} \, \frac{1}{\eta}.
\end{equation} 
Daily savings due to BESS is
\begin{equation}
    S_\text{v2} = \left(R_{\text{p}} - \frac{R_{\text{s}}}{\eta}\right) P_{\text{p}} T_{\text{p}} 
    = \left(R_{\text{p}} - \frac{R_{\text{s}}}{\eta}\right) \, 2 \, \eta \, C \, .
\end{equation}
The payback period is
\begin{equation}
    Y_\text{v2} = \frac{K \, C}{S_\text{v2} \, d_\text{y}} 
    = \frac{K}{2 \, \eta \, d_\text{y}} \frac{1}{\left(R_{\text{p}} - \frac{R_{\text{s}}}{\eta}\right)}.
\end{equation}
Note that the expression for the paypack period is independent of the
actual consumption, total captital cost, and capacity of the BESS. It
only depends on the capital cost per kWh, effective solar rate,
the round-trip efficiency, and the peak rate (in contrast to 
Model v1, which depended on normal rate).


\paragraph{Example 2.1} Payback period for a BESS with solar charging. 
Let $R_{\text{s}} = 2$ Rs./kWh, and the rest same as Example 1.1.
The payback period is
\begin{equation}
    Y_\text{v2} = \frac{30000}{2 \times 0.95 \times 300} 
    \frac{1}{8.75 - 2/0.95} \approx 8 \text{ years}.
\end{equation}

\section{Off-Grid, Solar-BESS Model v3}
This is an extreme case of assuming unlimited area available to
install solar panels so that it is not necessary (or possible) to use
Grid energy at all, eg. as in a remote location. Let $E_\text{sp}$ be the
average energy available from one solar panel per day and $N_\text{sp}$ be
the number of solar panels installed. The duration of solar availability
is $T_\text{s}$ hr per day.

\paragraph{Assumptions}
\begin{enumerate}
    \item Though there is no peak and normal load or rates in an off-grid strategy,
    we will use the same notation as in the previous models to obtain the
    load profile and compute the savings in comparison to the current electricity bill
    ($B_\text{now}$). 
    \item The day is divided into: Half peak load during solar availability,
    normal load during solar availability, 
    half peak load during night, normal load during night time, and an idle time 
    when there is no significant load ($P_\text{n} \gg P_\text{idle}\approx 0$).
    \begin{equation}
        24 = \left[ \frac{T_\text{p}}{2} + T_\text{nd} \right]_\text{day} 
        + \left[ \frac{T_\text{p}}{2} + T_\text{nn} + T_\text{idle} \right]_\text{night}
    \end{equation}
    \item During day time BESS is charged. This means solar
    energy is used for the load as well as charging BESS sufficiently
    for the night time discharge.

\end{enumerate}

\paragraph{Capacity of the BESS system} We have to revise the capacity from the
earlier estimate as the BESS now has to support a larger load.  BESS
discharges only during the night time supporting the half peak load and normal night load. 

\begin{equation}
    C = \frac{P_\text{p} T_\text{p} }{2 \eta} + \frac{P_\text{n} T_\text{nn} }{\eta}.
\end{equation}
During the day time, solar supports half of "peak" load, normal day-time load,
and charges the BESS.  Solar capacity needed is
\begin{equation}
    \label{eq:cs}
   C_\text{solar} = 
        \frac{P_\text{p} T_\text{p} }{2} + P_\text{n}  \left[T_\text{s} - \frac{T_\text{p}}{2} \right] + C
\end{equation}
The number of solar panels required is
\begin{equation}
    N_\text{sp} = \frac{C_\text{solar}}{E_\text{sp}}.
\end{equation}
The daily electricity bill for the off-grid system is
\begin{equation}
    B_\text{v3} = R_\text{s} \, C_\text{solar}.
\end{equation}
The savings due to the Solar-BESS is
\begin{equation}
    S_\text{v3} = B_\text{now} - B_\text{v3} 
\end{equation}
There is no simple expression for the payback period in this case.
\begin{equation}
    Y_\text{v3} = \frac{K \, C}{S_\text{v3} \, d_\text{y}}.
\end{equation}
and the payback period is to be calculated numerically step-wise, dependent
on all the parameters used BESS capital cost, power rates, load 
consumption pattern, solar availability, and round-trip efficiency.

\paragraph{Example 3.1} Off-grid Solar-BESS system.
Let $R_{\text{s}} = 2$ Rs./kWh, $R_\text{n} = 7$Rs./kWh, $\alpha=0.25$,
$T_\text{p}=8$~hrs, $P_\text{p} = 40$~kW, $T_\text{nd}=8$~hrs, $T_\text{nn}=0$,
$P_\text{n} = 50$~kW, $T_{\text{s}} = 7$~hrs, $E_{\text{sp}} = 4.5$ kWh, $K =
30000$ Rs./kWh, $\eta = 0.95$. 

The daily load is
\begin{align}
    E_{\text{load}} & = P_{\text{p}} T_{\text{p}} + P_{\text{n}} T_{\text{nd}} + P_\text{n} T_\text{nn} \\
    & = 40 \times 8 + 50 \times 8 + 0 = 720 \text{ kWh}.  
\end{align}
Energy bill without solar+BESS is
\begin{align}
    B_{\text{now}} & = 
    R_\text{p} \, P_{\text{p}} T_{\text{p}} + R_\text{n} ( 
    P_{\text{n}} T_{\text{nd}}+ P_\text{n} T_\text{nn} )   \\
    &= 8.75 \times 320 + 7 \times 400 = \text{Rs. } 5600
\end{align}
BESS capacity is
\begin{align}
    C & = \frac{P_\text{p} T_\text{p} }{2 \eta} + \frac{P_\text{n} T_\text{nn} }{\eta} \\
    & = \frac{40 \times 8}{2 \times 0.95} + 0 = 168 \text{ kWh} \approx\to 180 \text{ kWh}.
\end{align}
Using battery modules in units of approx 15 kWh, we need ceil(168/15) = 12 modules, or 180 kWh.
The solar capacity is
\begin{align}
    C_{\text{solar}} & = 
    \frac{P_{\text{p}} T_{\text{p}} }{2} + P_{\text{n}} \left[T_{\text{s}} - \frac{T_{\text{p}}}{2} \right] + C
    = 40 \times 4 + 50 \times 3 + 180 = 490 \text{ kWh}.
\end{align}
Number of solar panels required is
\begin{equation}
    N_{\text{sp}} = \frac{490}{4.5} \approx 109.
\end{equation}
Daily energy bill with Solar-BESS is
\begin{equation}
    B_{\text{v3}} = 2 \times 490 = 980 \text{ Rs.}
\end{equation}
Daily savings is
\begin{equation}
    S_{\text{v3}} = 5600 - 980 = 4620 \text{ Rs.}
\end{equation}
Capital cost of the BESS is
\begin{equation}
    K = 30000 \text{ Rs./kWh} \times 180 \text{ kWh} = \text{Rs. }54 \text{ lakhs}.
\end{equation}
Payback period is
\begin{equation}
    Y_{\text{v3}} = \frac{54 \text{ lakhs}}{4620 \times 300} \approx 4 \text{ years}.
\end{equation}

\section{Limited Solar-BESS Model v4}
In this model, the solar capacity is limited, and the BESS is charged
using solar power during the day, and discharged during the peak-rate hours.
Grid power is available to supplement the solar power for the load during 
peak hours, and for charging BESS during night time.


\section{Combustion Engine (Diesel generator) replacement BESS Model v5}

In this model, the BESS is used to replace a diesel generator or any other
fuel-combustion engine such as that using LNG. For simplicity, we will
assume that the generator is used only during the peak hours, when there
is likely to be a power cut. BESS is used to supply the load during the
peak hours and is charged during the normal hours (BESS peak shifting).

Let $R_{\text{g}} = (1 + \gamma) \, R_\text{n}$ be the rate (Rs./kWh) for the generator fuel, and
$T_{\text{g}}$ be the duration of the generator operation per day. The daily
electricity bill for the generator
supported backup power for peak hours is
\begin{equation}
    B_{\text{g}} = 
    R_{\text{n}} P_{\text{n}} T_{\text{n}} +
    R_{\text{p}} P_{\text{p}} (T_{\text{p}} - T_\text{g})  + 
    R_{\text{g}} P_{\text{p}} T_{\text{g}} .
\end{equation}
Replacing the generator with BESS during the peak hours,
and charging during the normal hours, the daily electricity bill is
\begin{equation}
    B_{\text{v5}} = 
    R_\text{n} \left( P_{\text{n}} T_{\text{n}} +  
    C \right)
\end{equation}
where $C$ is the capacity of the BESS required to support the peak power demand, 
and is same as in Model v1, Eq.~\eqref{eq:cs}.
The savings due to the BESS peak-shifting is
\begin{equation}
    S_\text{v5} = B_{\text{g}} - B_{\text{v5}} 
    = P_{\text{p}} T_{\text{g}} \left(R_{\text{g}} - C \right).
\end{equation}









\end{document}

%%% Local Variables:
%%% mode: latex
%%% TeX-master: t
%%% End:
