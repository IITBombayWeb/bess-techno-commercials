\section{BESS Peak-shifting Model v1}

\paragraph{Peak-shifting} This is the most common strategy for BESS
and means that the battery is charged during off-peak hours and 
discharged during peak hours. The objective is to reduce the peak
power demand and thereby reduce the electricity bill.

\paragraph{Assumptions}
\begin{enumerate}
    \item Grid power is available at two prices: $R_{\text{p}}$ for peak
    rate (Rs./kWh) and $R_{\text{n}}$ (Rs./kWh) for normal or base rate.
    The relationship is defined
    \begin{equation}
        R_{\text{p}} = (1 + \alpha) R_{\text{n}},
    \end{equation} 
    where $\alpha$ is the premium for peak-rate power.

    \item The peak-rate period per day is $T_{\text{p}}$ hours, and the 
    normal-rate duration is $T_{\text{n}}$ hours per day.
    $T_{\text{n}} > 2 T_{\text{p}} $, so that there is sufficient time 
    to charge the battery at 0.5~C at normal-rate. 

    \item During peak-rate hours, the power demand is $P_{\text{p}}$ kW, 
    and during normal-rate hours, the power demand is $P_{\text{n}}$ kW.

    \item The BESS has a discharge capacity of $C$ kWh, and the round-trip efficiency 
    is $\eta$, defined by
    \begin{equation}
        \eta = \frac{\text{Energy out}}{\text{Energy in}} = \frac{C}{C_{\text{in}}}.
    \end{equation}
    The capacity of the BESS is chosen such that the peak power demand is met. 
    The capacity required is
    \begin{equation}
        \label{eq:cv1}
        C = \frac{P_{\text{p}} T_{\text{p}} }{2}.
    \end{equation}
    The factor 2 accounts for the fact that the battery charged/discharged twice a day,
    because the peak-rate period is spaced out over two parts of the day.


\end{enumerate}

\paragraph{Model}

Without the BESS, the daily electricity bill (in Rs.) is
\begin{equation}
    B_{\text{now}} = 
    R_{\text{n}} P_{\text{n}} T_{\text{n}}  +
    R_{\text{p}} P_{\text{p}} T_{\text{p}} 
\end{equation}
With the BESS in peak-shifting mode, grid is used only during the normal-rate
hours, to supply power to the load as well as to charge the battery twice (up to
the capacity requried for the peak discharge).  The daily electricity bill is
\begin{equation}
    \label{eq:bv1}
    B_{\text{v1}} = 
    R_\text{n} \left(P_{\text{n}}  T_{\text{n}}  +  
     2 \, \frac{C}{\eta} \right)
\end{equation}
Daily savings due to BESS is
\begin{align}
    S_\text{v1} & = B_{\text{now}} - B_{\text{v1}} \\
    & =  R_{\text{p}} P_{\text{p}} T_{\text{p}} - 2 \, R_\text{n} \, C \\
    & =  R_{\text{n}} \, (1+ \alpha) 2 \, C - 2 \, R_\text{n} \, C/\eta  \\
    & =   2 \, R_{\text{n}} \, C \left[   (1 + \alpha) - 1/\eta \right].
    \label{eq:sv1}
\end{align}

\paragraph{Payback period} The capital cost of the BESS is $K$ Rs./kWh
of capacity. The payback period is the time taken for the savings to
equal the capital cost. The payback period (in years) is
\begin{equation}
    \label{eq:pb1}
    Y_\text{v1} = \frac{K \, C}{S_\text{v1} \, d_\text{y}} 
    = \frac{K}{2 \, R_\text{n} \, d_\text{y}} \frac{1}{\left[  (1 + \alpha)
    - 1/\eta\right]},
\end{equation} 
where $d_\text{y}$ is the number of active power consumption days in a year. 
Note that the expression for the paypack period is independent of the
actual consumption, total captital cost, and capacity of the BESS. It
only depends on the capital cost per kWh, premium for peak-rate power,
the round-trip efficiency, and the normal rate.

\paragraph{Example 1.1} Payback period for a BESS.
Let $R_{\text{n}} = 7$ Rs./kWh, $\alpha = 0.25$,
$d_\text{y} = 300$ days, $K = 30000$ Rs./kWh, $\eta = 0.90$.
The payback period directly evaluated from Eq.~\eqref{eq:pb1} is
\begin{equation}
    Y_\text{v1} = \frac{30000}{2 \times 7 \times 300} 
    \frac{1}{ [1.25 - 1/0.95]} \approx 38 \text{ years}.
\end{equation}
Additionally, to obtain the capacity of the BESS, let
$T_{\text{p}} = 8$ hours, $P_{\text{p}} = 50$~kW. From Eq.~\eqref{eq:cv1},
\begin{equation}
    C = \frac{50 \times 8}{2 \times 0.95} \approx 210 \text{ kWh}.
\end{equation}

\paragraph{Example 1.2} What does the peak-rate premium need to be for a payback period of 5 years. 
Approximating $\eta \approx 1$ in Eq.~\eqref{eq:pb1}, we have
\begin{equation}
    Y_\text{v1} = \frac{K}{  2 \, \alpha \, R_\text{n} \, d_\text{y}} 
\end{equation}
Or the premium should be
\begin{equation}
    \alpha = 
        \frac{K}{ 2\, Y_\text{v1} \, R_\text{n} \, d_\text{y}}
        = 
        \frac{30000}{2 \times 5 \times 7 \times 300} \approx 1.4
\end{equation}
The peak-rate should be 
\begin{equation}
    R_{\text{p}} = (1+1.4) \times 7 = 16.8 \text{ Rs./kWh}.
\end{equation}





% This is Silpa's first funded conference travel from IIT Bombay. Her performance so far is commendable.  I strongly recommend this support.


%%% Local Variables:
%%% mode: latex
%%% TeX-master: "paybackPeriod"
%%% End:
