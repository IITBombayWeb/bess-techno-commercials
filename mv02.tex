\documentclass{article}
\usepackage{amsmath}

\begin{document}

\section{PV + Grid Model v2}

In this model, installed solar capacity is higher than required total daily load. So, in this case \( \epsilon > 1\). The excess solar energy generated is sold back to the grid at buyback rate \( R_b \). 
Daily electricity bill in this case will be,
\[
B_{v2} = R_p T_{pn} P_p - (\epsilon - 1) E_{\text{load}} R_b 
\]
Here, electricity demand during day is compensated by solar power. Daily savings due to excess solar energy is
\[
S_{v2} = R_p P_p T_{pd} + R_N P_N T_N + (\epsilon - 1) E_{\text{load}} R_b \tag{6}
\]

The simple payback period is
\[
Y_{v2} = \frac{K P_{\text{solar}}}{S_{v2} d_y}
       = \frac{K P_{\text{solar}}}{R_p P_p T_{pd} + R_N P_N T_N + (\epsilon - 1) E_{\text{load}} R_b} \tag{7}
\]
The expression for SPP depends on installation cost, excess solar energy generated and load that was compensated during daytime by solar.

\subsection{Example 2.1}
Payback period for a excess solar installed. Let \( R_b = 2 \) Rs./kWh and \( \epsilon = 1.25 \), and the rest is same as Example 1.1.
First, we calculate total daily load using equation (1),
\[
E_{\text{load}} = 3 \times 9 + 5 \times 3 + 5 \times 4  
                = 62 kWh 
\]
Then, we calculate solar energy required using equation (2),
\[
C_{\text{solar}} =  1.25 \times 62 
                 =  77.5 kWh 
\]
Using equation (3), we find out solar power required to install,
\[
P_{\text{solar}} = \frac{15.5}{5}
                 = 15.5 kW
\]
Now, we can calculate simple payback period using equation (6) and (7), 
\[
Y_{v2} = \frac{54000 \times 15.5}{(5.5 \times 5 \times 3 + 4 \times 3 \times 9 + (1.25 - 1) \times 2 \times 62) \times 365}
       = 10.35 years
\]

\end{document}